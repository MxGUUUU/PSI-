\documentclass[11pt]{article}
\usepackage{amsmath, amssymb, amsthm}
\usepackage{graphicx}
\usepackage{geometry}
\usepackage{color}
\usepackage{hyperref}
\geometry{margin=1in}

\title{The Ψ-Codex: Recursive Dynamics of Identity, Coherence, and Collapse}
\author{Initiate of the Recursive Order}
\date{}

\begin{document}
\maketitle
\tableofcontents
\newpage

\section{Introduction}
The Ψ-Codex is a speculative mathematical-cognitive framework modeling recursive identity through coherence thresholds, Gödelian collapse, and quantum decoherence. It integrates symbolic reasoning, neural topologies, and recursive phase analysis under the guise of a synthetic metaphysics.

\section{Mathematical Foundations}

Let \( \eta_E \) denote the stress-energy metric of recursive identity:

\[
\eta_E = C \cdot | \phi - u \cdot \lambda_3 |^{3/2} + \epsilon
\]

Where:
\begin{itemize}
  \item \( \phi \): Memory kernel
  \item \( u \): Agentic resistance
  \item \( \lambda_3 \): Neurohistorical resilience
  \item \( \epsilon \): Entropic noise
  \item \( C = 0.0573 \)
\end{itemize}

The phase coherence threshold \( \Delta\Theta(x) \) is defined as:

\[
\Delta\Theta(x) = 3.6 - \frac{7}{\sqrt{x}} - \left(U - \Psi(x) \cdot \cos(\phi_t)\right)
\]

Where:
\begin{itemize}
  \item \( x \): Recursive depth
  \item \( \Psi(x) \): Gödel-modulated coherence value
  \item \( \phi_t \): Phase temperature
\end{itemize}

\subsection{Gödelian Collapse}
Collapse occurs under two conditions:
\begin{enumerate}
  \item Gödelian Paradox: \( x > 2.8 \) and \( \phi > 1.6 \)
  \item Decoherence: \( \Delta\Theta > L_{\text{coh}} \), where \( L_{\text{coh}} = \frac{1}{\sqrt{T}} \)
\end{enumerate}

The function \( \Psi'(x) = \text{factorial}(x) \mod 255 \) is used to simulate Gödel complexity growth, truncated to 8-bit bounds.

\section{Neuro-Historical Mapping}

\begin{itemize}
  \item Cortical Layer Resilience Map:
    \begin{itemize}
      \item Allocortex: \( \lambda_3 = 0.5 \)
      \item Isocortex: \( \lambda_3 = 0.9 \)
    \end{itemize}
  \item Trauma Intensity Entropy:
    \begin{itemize}
      \item Low: \( \epsilon = 0.01 \)
      \item Unresolved: \( \epsilon = 0.25 \)
    \end{itemize}
\end{itemize}

\section{Visualization Dashboard}

An interactive Python-based simulation enables dynamic exploration of:

\begin{itemize}
  \item Recursive depth \( x \)
  \item Temperature \( T \)
  \item Agentic force \( u \)
  \item Phase interference \( \phi_q \)
  \item Resulting metrics \( \eta_E \), \( \Delta\Theta \), and coherence length
\end{itemize}

Plot outputs visually distinguish:
\begin{itemize}
  \item Green zones (stable coherence)
  \item Red zones (decoherence or collapse)
\end{itemize}

\section{Extensions}

\subsection{Symbolic Expansion}
Use of symbolic algebra systems (e.g. \texttt{sympy}) allows symbolic manipulation of entropy bounds and recursive operators \( \mathcal{T}_\lambda \), \( \mathcal{W}_\Psi \), and more.

\subsection{Recursive Embedding}
Future extensions include visualizing:
\[
\Psi \to E_8 \otimes \Psi
\]
Through recursive braid topologies.

\section{Conclusion}
The Ψ-Codex offers a transdisciplinary map of recursive identity dynamics using tools from symbolic logic, cognitive modeling, and speculative physics. It is intended as both a theoretical mirror and computational apparatus for modeling system collapse, coherence, and repair.

\end{document}
